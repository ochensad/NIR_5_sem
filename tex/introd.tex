
\chapter*{Реферат}
\addcontentsline{toc}{chapter}{Реферат}

Целью данной научно-исследовательской работы является классификация существующих методов многокритериального выбора.

В результате были рассмотрены математические основы теории принятия решений, проведен обзор многокритериальных задач и методов их решения. Также была проведена классификация рассмотренных методов на основе определенного в работе критерия, были предложены наиболее подходящие области применения данных методов.

Ключевые слова: теория принятия решений, многокритериальные задачи, методы многокритериального выбора, классификация, обзор многокритериальных задач.

\newpage

\chapter*{Введение}
\addcontentsline{toc}{chapter}{Введение}

Человек, по мере своего существования постоянно сталкивается с ситуациями, требующими осуществления выбора. Например, приходя в кофейню мы выбираем тот или иной кофе, в зависимости от своих предпочтений. Чтобы добраться до места назначения мы выбираем маршрут и соответствующий вид транспорта. А при поиске работы мы выбираем место, на котором хотели бы работать.

Задачи выбора, в которых преследуются сразу несколько целей, называются многокритериальными. Именно с такими задачами чаще всего приходится сталкиваться человеку в своей деятельности. Действительность такова, что на данный момент не существует единого метода для решения многокритериальных задач. Решение таких задач в большей степени зависит от четкости понимания всех целей, лимита времени на процесс принятия решений и от многого другого. \cite{bib1}

Целью данной научно-исследовательской работы является классификация существующих методов многокритериального выбора. Для достижения поставленной цели необходимо решить следующие задачи:
\begin{itemize}[label=--]
    \item рассмотреть математические основы теории принятия решений;
    \item провести обзор многокритериальных задач и методов их решения;
    \item определить критерий классификации методов многокритериального выбора;
    \item классифицировать методы многокритериального выбора;
    \item предложить наиболее подходящие области применения рассмотренных методов.
\end{itemize}